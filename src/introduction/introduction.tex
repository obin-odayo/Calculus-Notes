% Notes on Differential Calculus
% By: Zhean Robby L. Ganituen (Obin Odayo)
% Created on December 2023
% Introduction Chapter

\chapter{Introduction to Differential Calculus}

\section{Functions}

A function has some input and exactly one unique output. Therefore, all functions have \emph{one-to-one correspondence}. In the expression below the function is denoted as \(f\), and the input is denoted as \((x)\) and the output is \(y\).

\[
    f(x) = y
\]

Since all functions are one-to-one, therefore:

\begin{align*}
    f(x_1) &= y_1 \\ 
    f(x_2) &= y_2 \\
           &\vdots \\
    f(x_n) &= y_n \\
\end{align*}

For \(y_1, y_2, \cdots, y_n\) are unique outputs for \(x_1, x_2, \cdots, x_n\).

\begin{ex}
    \label{ex1}
    Functions can be used in the weight of fishes.
\end{ex}

\begin{center}
    \begin{tabular}{cc}
        \toprule
        Fish & Mass \\
        \midrule
        1 & 50 \\
        2 & 33 \\
        3 & 100 \\
        \bottomrule
    \end{tabular}
\end{center}

Notice in \cref{ex1}, each fish has a corresponding mass. Also notice how the mass of each fish is very unique to itself, only Fish 1 can have the mass of 33. This is how functions work. The numbers corresponding to the fish are the inputs, while the masses of each fish are the output.

Therefore, we can express
